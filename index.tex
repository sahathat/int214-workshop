% Options for packages loaded elsewhere
\PassOptionsToPackage{unicode}{hyperref}
\PassOptionsToPackage{hyphens}{url}
%
\documentclass[
]{article}
\usepackage{amsmath,amssymb}
\usepackage{lmodern}
\usepackage{ifxetex,ifluatex}
\ifnum 0\ifxetex 1\fi\ifluatex 1\fi=0 % if pdftex
  \usepackage[T1]{fontenc}
  \usepackage[utf8]{inputenc}
  \usepackage{textcomp} % provide euro and other symbols
\else % if luatex or xetex
  \usepackage{unicode-math}
  \defaultfontfeatures{Scale=MatchLowercase}
  \defaultfontfeatures[\rmfamily]{Ligatures=TeX,Scale=1}
\fi
% Use upquote if available, for straight quotes in verbatim environments
\IfFileExists{upquote.sty}{\usepackage{upquote}}{}
\IfFileExists{microtype.sty}{% use microtype if available
  \usepackage[]{microtype}
  \UseMicrotypeSet[protrusion]{basicmath} % disable protrusion for tt fonts
}{}
\makeatletter
\@ifundefined{KOMAClassName}{% if non-KOMA class
  \IfFileExists{parskip.sty}{%
    \usepackage{parskip}
  }{% else
    \setlength{\parindent}{0pt}
    \setlength{\parskip}{6pt plus 2pt minus 1pt}}
}{% if KOMA class
  \KOMAoptions{parskip=half}}
\makeatother
\usepackage{xcolor}
\IfFileExists{xurl.sty}{\usepackage{xurl}}{} % add URL line breaks if available
\IfFileExists{bookmark.sty}{\usepackage{bookmark}}{\usepackage{hyperref}}
\hypersetup{
  pdftitle={INT214: Workshop},
  pdfauthor={by: \_\_\_\_\_\_\_\_\_\_\_\_\_},
  hidelinks,
  pdfcreator={LaTeX via pandoc}}
\urlstyle{same} % disable monospaced font for URLs
\usepackage[margin=1in]{geometry}
\usepackage{color}
\usepackage{fancyvrb}
\newcommand{\VerbBar}{|}
\newcommand{\VERB}{\Verb[commandchars=\\\{\}]}
\DefineVerbatimEnvironment{Highlighting}{Verbatim}{commandchars=\\\{\}}
% Add ',fontsize=\small' for more characters per line
\usepackage{framed}
\definecolor{shadecolor}{RGB}{248,248,248}
\newenvironment{Shaded}{\begin{snugshade}}{\end{snugshade}}
\newcommand{\AlertTok}[1]{\textcolor[rgb]{0.94,0.16,0.16}{#1}}
\newcommand{\AnnotationTok}[1]{\textcolor[rgb]{0.56,0.35,0.01}{\textbf{\textit{#1}}}}
\newcommand{\AttributeTok}[1]{\textcolor[rgb]{0.77,0.63,0.00}{#1}}
\newcommand{\BaseNTok}[1]{\textcolor[rgb]{0.00,0.00,0.81}{#1}}
\newcommand{\BuiltInTok}[1]{#1}
\newcommand{\CharTok}[1]{\textcolor[rgb]{0.31,0.60,0.02}{#1}}
\newcommand{\CommentTok}[1]{\textcolor[rgb]{0.56,0.35,0.01}{\textit{#1}}}
\newcommand{\CommentVarTok}[1]{\textcolor[rgb]{0.56,0.35,0.01}{\textbf{\textit{#1}}}}
\newcommand{\ConstantTok}[1]{\textcolor[rgb]{0.00,0.00,0.00}{#1}}
\newcommand{\ControlFlowTok}[1]{\textcolor[rgb]{0.13,0.29,0.53}{\textbf{#1}}}
\newcommand{\DataTypeTok}[1]{\textcolor[rgb]{0.13,0.29,0.53}{#1}}
\newcommand{\DecValTok}[1]{\textcolor[rgb]{0.00,0.00,0.81}{#1}}
\newcommand{\DocumentationTok}[1]{\textcolor[rgb]{0.56,0.35,0.01}{\textbf{\textit{#1}}}}
\newcommand{\ErrorTok}[1]{\textcolor[rgb]{0.64,0.00,0.00}{\textbf{#1}}}
\newcommand{\ExtensionTok}[1]{#1}
\newcommand{\FloatTok}[1]{\textcolor[rgb]{0.00,0.00,0.81}{#1}}
\newcommand{\FunctionTok}[1]{\textcolor[rgb]{0.00,0.00,0.00}{#1}}
\newcommand{\ImportTok}[1]{#1}
\newcommand{\InformationTok}[1]{\textcolor[rgb]{0.56,0.35,0.01}{\textbf{\textit{#1}}}}
\newcommand{\KeywordTok}[1]{\textcolor[rgb]{0.13,0.29,0.53}{\textbf{#1}}}
\newcommand{\NormalTok}[1]{#1}
\newcommand{\OperatorTok}[1]{\textcolor[rgb]{0.81,0.36,0.00}{\textbf{#1}}}
\newcommand{\OtherTok}[1]{\textcolor[rgb]{0.56,0.35,0.01}{#1}}
\newcommand{\PreprocessorTok}[1]{\textcolor[rgb]{0.56,0.35,0.01}{\textit{#1}}}
\newcommand{\RegionMarkerTok}[1]{#1}
\newcommand{\SpecialCharTok}[1]{\textcolor[rgb]{0.00,0.00,0.00}{#1}}
\newcommand{\SpecialStringTok}[1]{\textcolor[rgb]{0.31,0.60,0.02}{#1}}
\newcommand{\StringTok}[1]{\textcolor[rgb]{0.31,0.60,0.02}{#1}}
\newcommand{\VariableTok}[1]{\textcolor[rgb]{0.00,0.00,0.00}{#1}}
\newcommand{\VerbatimStringTok}[1]{\textcolor[rgb]{0.31,0.60,0.02}{#1}}
\newcommand{\WarningTok}[1]{\textcolor[rgb]{0.56,0.35,0.01}{\textbf{\textit{#1}}}}
\usepackage{graphicx}
\makeatletter
\def\maxwidth{\ifdim\Gin@nat@width>\linewidth\linewidth\else\Gin@nat@width\fi}
\def\maxheight{\ifdim\Gin@nat@height>\textheight\textheight\else\Gin@nat@height\fi}
\makeatother
% Scale images if necessary, so that they will not overflow the page
% margins by default, and it is still possible to overwrite the defaults
% using explicit options in \includegraphics[width, height, ...]{}
\setkeys{Gin}{width=\maxwidth,height=\maxheight,keepaspectratio}
% Set default figure placement to htbp
\makeatletter
\def\fps@figure{htbp}
\makeatother
\setlength{\emergencystretch}{3em} % prevent overfull lines
\providecommand{\tightlist}{%
  \setlength{\itemsep}{0pt}\setlength{\parskip}{0pt}}
\setcounter{secnumdepth}{-\maxdimen} % remove section numbering
\ifluatex
  \usepackage{selnolig}  % disable illegal ligatures
\fi

\title{INT214: Workshop}
\author{by: \_\_\_\_\_\_\_\_\_\_\_\_\_}
\date{}

\begin{document}
\maketitle

\hypertarget{r-markdown}{%
\subsection{R Markdown}\label{r-markdown}}

This is an R Markdown document. Markdown is a simple formatting syntax
for authoring HTML, PDF, and MS Word documents. For more details on
using R Markdown see \url{http://rmarkdown.rstudio.com}.

When you click the \textbf{Knit} button a document will be generated
that includes both content as well as the output of any embedded R code
chunks within the document. You can embed an R code chunk like this:

\hypertarget{import-library-and-preparing-dataset}{%
\subsubsection{1.) Import Library and Preparing
Dataset}\label{import-library-and-preparing-dataset}}

\hypertarget{import-lib}{%
\paragraph{Import Lib}\label{import-lib}}

\begin{Shaded}
\begin{Highlighting}[]
\FunctionTok{library}\NormalTok{(readr)}
\FunctionTok{library}\NormalTok{(dplyr)}
\FunctionTok{library}\NormalTok{(ggplot2)}
\FunctionTok{library}\NormalTok{(rmarkdown)}
\end{Highlighting}
\end{Shaded}

\hypertarget{import-dataset}{%
\paragraph{Import Dataset}\label{import-dataset}}

\begin{Shaded}
\begin{Highlighting}[]
\NormalTok{int214 }\OtherTok{\textless{}{-}} \FunctionTok{read\_csv}\NormalTok{(}\StringTok{"db\_int214.csv"}\NormalTok{)}
\end{Highlighting}
\end{Shaded}

\begin{verbatim}
## Rows: 140 Columns: 13
\end{verbatim}

\begin{verbatim}
## -- Column specification --------------------------------------------------------
## Delimiter: ","
## chr (9): timestamp, skill_excel, skill_r, skill_stat, music_genres, sec, gen...
## dbl (4): int214_level, int214_att, std_ready, salary
\end{verbatim}

\begin{verbatim}
## 
## i Use `spec()` to retrieve the full column specification for this data.
## i Specify the column types or set `show_col_types = FALSE` to quiet this message.
\end{verbatim}

\hypertarget{explore-data}{%
\subsubsection{2.) Explore Data}\label{explore-data}}

\begin{Shaded}
\begin{Highlighting}[]
\NormalTok{int214 }\SpecialCharTok{\%\textgreater{}\%} \FunctionTok{glimpse}\NormalTok{()}
\end{Highlighting}
\end{Shaded}

\begin{verbatim}
## Rows: 140
## Columns: 13
## $ timestamp    <chr> "5/8/2021, 0:57:41", "4/8/2021, 23:45:05", "9/8/2021, 13:~
## $ skill_excel  <chr> "Occasionally", "Occasionally", "Occasionally", "Occasion~
## $ skill_r      <chr> "Never used", "Never used", "Occasionally", "Never used",~
## $ skill_stat   <chr> "Level 0", "Level 1", "Level 2", "Level 2", "Level 1", "L~
## $ int214_level <dbl> 5, 3, 3, 2, 3, 4, 5, 4, 4, 4, 3, 4, 5, 5, 5, 4, 3, 5, 5, ~
## $ int214_att   <dbl> 5, 3, 4, 5, 4, 3, 2, 5, 4, 3, 4, 4, 4, 4, 3, 4, 4, 3, 4, ~
## $ std_ready    <dbl> 5, 3, 4, 5, 3, 3, 2, 5, 3, 4, 5, 4, 4, 4, 4, 3, 3, 4, 4, ~
## $ music_genres <chr> "Pop, Rock, R&B", "Pop, Rock, R&B, Rap", "R&B, Rap", "Pop~
## $ salary       <dbl> 80000, 25000, 40000, 35000, 40000, 30000, 45000, 20000, 3~
## $ sec          <chr> "Sec B", "Sec A", "Sec A", "Sec B", "Sec A", "Sec B", "Se~
## $ gender       <chr> "female", "LGBTQ+", "male", "female", "female", "female",~
## $ birthday     <chr> "27/9/2001", "20/5/2002", "31/8/2001", "7/5/2002", "12/8/~
## $ program      <chr> "Arts-Math", "Arts-Math", "Sci-Math", "Sci-Math", "Sci-Ma~
\end{verbatim}

\begin{Shaded}
\begin{Highlighting}[]
\NormalTok{int214 }\SpecialCharTok{\%\textgreater{}\%} \FunctionTok{summary}\NormalTok{()}
\end{Highlighting}
\end{Shaded}

\begin{verbatim}
##   timestamp         skill_excel          skill_r           skill_stat       
##  Length:140         Length:140         Length:140         Length:140        
##  Class :character   Class :character   Class :character   Class :character  
##  Mode  :character   Mode  :character   Mode  :character   Mode  :character  
##                                                                             
##                                                                             
##                                                                             
##   int214_level    int214_att      std_ready     music_genres      
##  Min.   :1.00   Min.   :1.000   Min.   :1.000   Length:140        
##  1st Qu.:3.00   1st Qu.:3.000   1st Qu.:3.000   Class :character  
##  Median :4.00   Median :3.000   Median :3.000   Mode  :character  
##  Mean   :3.85   Mean   :3.571   Mean   :3.436                     
##  3rd Qu.:5.00   3rd Qu.:4.000   3rd Qu.:4.000                     
##  Max.   :5.00   Max.   :5.000   Max.   :5.000                     
##      salary             sec               gender            birthday        
##  Min.   :    9000   Length:140         Length:140         Length:140        
##  1st Qu.:   25000   Class :character   Class :character   Class :character  
##  Median :   30000   Mode  :character   Mode  :character   Mode  :character  
##  Mean   :  108336                                                           
##  3rd Qu.:   40000                                                           
##  Max.   :10000000                                                           
##    program         
##  Length:140        
##  Class :character  
##  Mode  :character  
##                    
##                    
## 
\end{verbatim}

\begin{Shaded}
\begin{Highlighting}[]
\CommentTok{\# Sec}
\NormalTok{int214 }\SpecialCharTok{\%\textgreater{}\%} \FunctionTok{group\_by}\NormalTok{(sec) }\SpecialCharTok{\%\textgreater{}\%} \FunctionTok{count}\NormalTok{()}
\end{Highlighting}
\end{Shaded}

\begin{verbatim}
## # A tibble: 2 x 2
## # Groups:   sec [2]
##   sec       n
##   <chr> <int>
## 1 Sec A    71
## 2 Sec B    69
\end{verbatim}

\begin{Shaded}
\begin{Highlighting}[]
\CommentTok{\# Excel Skill}
\NormalTok{int214 }\SpecialCharTok{\%\textgreater{}\%} \FunctionTok{group\_by}\NormalTok{(skill\_excel) }\SpecialCharTok{\%\textgreater{}\%} \FunctionTok{count}\NormalTok{()}
\end{Highlighting}
\end{Shaded}

\begin{verbatim}
## # A tibble: 3 x 2
## # Groups:   skill_excel [3]
##   skill_excel      n
##   <chr>        <int>
## 1 Never used      27
## 2 Occasionally   109
## 3 Usually          4
\end{verbatim}

\begin{Shaded}
\begin{Highlighting}[]
\CommentTok{\# R Skill}
\NormalTok{int214 }\SpecialCharTok{\%\textgreater{}\%} \FunctionTok{group\_by}\NormalTok{(skill\_r) }\SpecialCharTok{\%\textgreater{}\%} \FunctionTok{count}\NormalTok{()}
\end{Highlighting}
\end{Shaded}

\begin{verbatim}
## # A tibble: 2 x 2
## # Groups:   skill_r [2]
##   skill_r          n
##   <chr>        <int>
## 1 Never used     137
## 2 Occasionally     3
\end{verbatim}

\hypertarget{data-visualization}{%
\subsubsection{3.) Data Visualization}\label{data-visualization}}

You can also embed plots, for example:

\begin{Shaded}
\begin{Highlighting}[]
\CommentTok{\# Example 1}
\NormalTok{int214 }\SpecialCharTok{\%\textgreater{}\%} \FunctionTok{ggplot}\NormalTok{(}\FunctionTok{aes}\NormalTok{(}\AttributeTok{x =}\NormalTok{ sec)) }\SpecialCharTok{+} \FunctionTok{geom\_bar}\NormalTok{(}\FunctionTok{aes}\NormalTok{(}\AttributeTok{fill=}\NormalTok{program)) }\SpecialCharTok{+} \FunctionTok{ggtitle}\NormalTok{(}\StringTok{"Example 1: Bar Chart"}\NormalTok{)}
\end{Highlighting}
\end{Shaded}

\includegraphics{index_files/figure-latex/plot1-1.pdf}

\begin{Shaded}
\begin{Highlighting}[]
\CommentTok{\# Example 2}
\NormalTok{int214 }\SpecialCharTok{\%\textgreater{}\%} \FunctionTok{filter}\NormalTok{(salary}\SpecialCharTok{\textless{}}\DecValTok{150000}\NormalTok{) }\SpecialCharTok{\%\textgreater{}\%} \FunctionTok{ggplot}\NormalTok{(}\FunctionTok{aes}\NormalTok{(}\AttributeTok{x=}\NormalTok{salary)) }\SpecialCharTok{+} \FunctionTok{geom\_histogram}\NormalTok{(}\AttributeTok{bins =} \DecValTok{10}\NormalTok{,}\AttributeTok{fill=}\StringTok{"\#69b3a2"}\NormalTok{,}\AttributeTok{color=}\StringTok{"\#e9ecef"}\NormalTok{) }\SpecialCharTok{+} \FunctionTok{ggtitle}\NormalTok{(}\StringTok{"Example 2: Expected Salary (THB)"}\NormalTok{)}
\end{Highlighting}
\end{Shaded}

\includegraphics{index_files/figure-latex/plot2-1.pdf}

\begin{Shaded}
\begin{Highlighting}[]
\CommentTok{\# Example 3}
\NormalTok{sample\_data }\OtherTok{\textless{}{-}} \FunctionTok{data.frame}\NormalTok{(}\AttributeTok{height =} \FunctionTok{rnorm}\NormalTok{(}\DecValTok{140}\NormalTok{,}\AttributeTok{mean=}\DecValTok{165}\NormalTok{, }\AttributeTok{sd=}\FloatTok{3.5}\NormalTok{))}
\FunctionTok{ggplot}\NormalTok{(sample\_data, }\FunctionTok{aes}\NormalTok{(}\AttributeTok{x=}\NormalTok{height)) }\SpecialCharTok{+} \FunctionTok{geom\_density}\NormalTok{(}\AttributeTok{fill=}\StringTok{"\#69b3a2"}\NormalTok{,}\AttributeTok{color=}\StringTok{"\#e9ecef"}\NormalTok{) }\SpecialCharTok{+} 
  \FunctionTok{ggtitle}\NormalTok{(}\StringTok{"Example 3: Students Height"}\NormalTok{)}
\end{Highlighting}
\end{Shaded}

\includegraphics{index_files/figure-latex/plot3-1.pdf}

\hypertarget{convert-file-to-html}{%
\subsubsection{4.) Convert file to HTML}\label{convert-file-to-html}}

\begin{itemize}
\tightlist
\item
  Click button \texttt{Knit} in R Studio Desktop to save in HTML file.
  Then, push this file to your github
\end{itemize}

\begin{center}\rule{0.5\linewidth}{0.5pt}\end{center}

Created Template File by
\href{https://github.com/safesit23}{Safe\_SIT23}

\end{document}
